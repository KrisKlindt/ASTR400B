\documentclass[trackchanges, twocolumn]{aastex7}

\newcommand{\vdag}{(v)^\dagger}
\newcommand\aastex{AAS\TeX}
\newcommand\latex{La\TeX}


\begin{document}

\title{Tidal Transformations of M33 During the Merging of the Milky Way and M31}

\author{Kris Klindt}
\affiliation{University of Arizona}
\email{kklindt@arizona.edu}

\begin{abstract}
    Tidal transformations are the changes to a satellite galaxy due to the tides from a massive host galaxy, such as changes in the internal stellar kinematics of the satellite galaxy. Understanding these transformations is important because of their implications on galaxy evolution, in particular how satellite galaxies evolve due to their interactions with their host galaxy. I will be using the N-body simulation of the merging of the Milky and M31 that includes M33 that is used in \cite{MarelBesla2012} as a tool to explore this topic. I will be trying to determine how the merger of the Milky Way and M31 will affect the evolution of M33, in particular, whether M33 will evolve from a dwarf irregular galaxy to a dwarf spheroidal/elliptical galaxy due to tidal transformations from the merging of the Milky Way and M31. I found that, as time passes, the difference between the analytical and mean circular velocity gets further from zero, which indicates that M33 is becoming less rotationally supported, in other words, less like a dwarf irregular. This shows that there is a correlation between tidal transformations and a satellite galaxy's evolution from a dwarf irregular to a dwarf spheroidal/elliptical galaxy.
\end{abstract}

\keywords{Local Group, Jacobi Radius, Tidal Stripping/Sharing, Satellite Galaxy, Rotation Curve}

\section{Introduction} 

Tidal transformations are an important factor in galaxy evolution, especially for satellite galaxies. One way to measure the tidal transformations of a satellite galaxy is by the changes to the internal stellar kinematics of the satellite galaxy due to the tides from a massive host galaxy. For example, as the satellite galaxy orbits around its host, mass can be stripped away from it, resulting in a loss of mass that affects quantities like the rotation curve of the satellite galaxy. 

In order to understand why tidal transformations are an important topic, one must know what galaxies and galaxy evolution are. According to Willman \& Strader, a \textbf{galaxy} is "a gravitationally bound set of stars whose properties cannot be explained by a combination of baryons (gas, dust, and stars) and Newton's laws of gravity" \citep{Willman2012}. \textbf{Galaxy evolution} is how the properties of a galaxy, such as color, brightness, size, and structure change with time, and why these changes occur. Tidal transformations are an important topic to our understanding of galaxy evolution for several reasons. One reason why the study of tidal transformations in satellite galaxies is important is that it gives us a better understanding of our local group of galaxies, in particular the satellites of the Milky Way and M31. These galaxies are made up of dwarf irregulars (disks, rotationally supported) and dwarf spheroidal (dispersion, no rotation) galaxies, and understanding the effects of tidal transformations on satellite galaxies can give us a connection between the two types of galaxies.

A particular pattern that has been noticed within the \textbf{Local Group}, which is made of the galaxies nearby the Milky Way that are gravitationally bound to each other (Besla, Class Lecture, 2025), is that most galaxies that are close to the Milky Way and M31 are either dwarf spheroidals (dSph’s) or dwarf ellipticals (dE's) and are mostly devoid of gas and supported by velocity dispersion \citep{Mayer2001}. Tidal transformations are a large factor as to why this is the case. The effect of mass being stripped on the \textbf{rotation curve}, which is the circular velocity of the stars in the galaxy as a function of radius from the galactic center, of satellite galaxies is also known. In their simulation of M33 and M31's most recent interaction, Semczuk et al. find the rotation curve shown in Figure \ref{M33RotationCurve}, which clearly demonstrates that the rotation curve of the satellite galaxy has a decrease in magnitude as the distance from the galaxy increases, since a lot of mass has been stripped from the satellite galaxy \citep{Semczuk2018}.

\begin{figure}[ht!] \label{M33RotationCurve}
\plotone{M33_RotationCurve.jpg}
\caption{ This is the simulated rotation curve of M33 (red, blue, and green lines) compared to the currently observed rotation curve (black line). This demonstrates that the rotation curve of the satellite galaxy has a decrease in magnitude as the distance from the galaxy increases \citep{Semczuk2018}.
\label{fig:general}}
\end{figure}

One of the open questions related to this topic is how the evolution of satellite galaxies is affected by their hosts. One example of this can be seen in Pardy et al.'s paper that analyzes offset stellar disks \citep{Pardy2016}. They analyze how tidal transformations can shift photometric galaxy centers off from the galaxy's dynamical centers. The varied effects that tidal transformations can have on a satellite galaxy are not fully understood, and there are many potentially surprising effects that tidal transformations can have on the evolution of a satellite galaxy.

\section{This Project} \label{sec:style}

The specific question I will be answering in this project will be how M33's analytical stellar rotation curve and mean stellar circular velocity evolve as the Milky Way and M31 merge.

This project will be addressing the question of how the evolution of satellite galaxies is affected by their hosts, including how the size of the host can change the affect on the satellite galaxy's evolution.

Many of the galaxies within our Local Group are satellite galaxies, and we can see that most galaxies that are close to the Milky Way and M31 are either dSph's or dE's \citep{Mayer2001}. This study aims to test if M33, which is currently a dwarf irregular, will evolve into a dSph due to tidal transformations from M31 and the Milky Way.
\newline
\section{Methodology}

I will be using the N-body simulation of the merging of the Milky and M31 that includes M33 that is used in \cite{MarelBesla2012}. An N-body simulation uses many (in this case, hundreds of thousands) point masses representing stars and dark matter and calculates the gravitational interactions between all those point masses in order to simulate how the galaxies evolve. The initial model of each galaxy uses a Hernquist profile \citep{Hernquist1990} for the dark matter halo, while the stellar bulge and disk of the simulation were modeled to match the surface-brightness profiles of each component of each galaxy. 

I will be using the VHighRes simulation to show how the rotation curve of M33 evolves with time. I will need to use all particle types in order to calculate the analytical rotation curve such as in Figure \ref{fig:Lab7}, and for the mean circular velocity, I will be using the disk particles.

\begin{figure}[ht!]
\plotone{Lab7_RotationCurve.png}
\caption{ This shows the analytical rotation curve of M31 (red line). The analytical rotation curves of M33 I will be calculating should be similar to this initially.
\label{fig:Lab7}}
\end{figure}

The code I will need to write will calculate the \textbf{Jacobi radius} of M33 using the equation: 
\(r_j = r*(M_{sat}/2M_{host})\),
where r is the distance between the centers of mass of M33 and M31 (kpc), \(M_{sat}\) is the mass of the satellite galaxy M33 (\(M_\odot\)), and \(M_{host}\) is the mass of the host galaxy, which will be either M31 or the combined Milky Way and M31 after they merge (\(M_\odot\)). I will find the analytical rotation curve for M33 within the Jacobi radius using the equation:
\(V_c^2=GM/r\),
where r is the galactocentric radius (kpc) and M is the mass of the galaxy enclosed within that radius (\(M_\odot\)). I will also need to convert the velocity to cylindrical coordinates in order to find the mean circular velocity \(v_\phi\) within a radial bin of 0.5 kpc going out to the Jacobi radius using the equations \(\rho=\sqrt{x^2+y^2}\) and \(v_\phi=(xv_y-yv_x)/\rho\), where x and y are the x and y components of each particle's distance from the center of mass of M33 (kpc) and \(v_x\) and \(v_y\) are the x and y components of the velocity of the particle. These directions will all be in a rotated frame, which is rotated so that the z-axis aligns with the disk's angular momentum.

In order to study how M33 evolves as the Milky Way and M31 merge, I will be using the steps outlined in Lab 7 in order to produce an analytical rotation curve for several pericenters and apocenters of M33's orbit around M31 and the combined Milky Way-M31 system such as what is shown in Figure \ref{fig:Lab7}. I will also calculate the actual mean circular velocity of the stars and take the difference between the resulting rotation curve and the analytical rotation curve. These plots will give an idea of how M33 loses mass as it orbits M31 as well as whether M33 is supported by rotation or by dispersion, which tells if it stays a dwarf irregular or becomes a dwarf spheroidal as the Milky Way and M31 merge.

I hypothesize that, as the Milky Way and M31 merge, some of the stars and dark matter particles will be torn from the outer edges of M33, causing the furthest (in distance) edges of the phase diagram to have lower velocities than initially, since there is less mass to sustain those high velocities. Also, I hypothesize that the plot of the difference between the actual mean circular velocity rotation curve and the analytical rotation curve will show that, as the Milky Way and M31 merge, the actual rotation curve will differ greatly from the analytical one, showing that M33 is more dispersion supported than rotationally supported. I believe this will happen because of the pattern that galaxies close to the Milky Way and M31 are dwarf spheroidal or dwarf ellipticals, which are dispersion supported, since M33's orbital distance decreases as the Milky Way and M31 merge.

\section{Results}

The analytical rotation curves for each apocenter and pericenter are shown in Figure \ref{fig:M33Rot}. The circular velocity beyond a certain point seems to decrease as time passes, which is indicative of the loss of M33's mass during its orbit.

The differences between the analytical and calculated circular velocities for each apocenter and pericenter are shown in Figure \ref{fig:M33Vphi}. If the calculated circular velocity is similar to the analytical one (implying M33 is rotationally supported), then the difference between the two should be close to zero. The beginning of the plot being in the negatives is due to the analytical velocity starting at 0, which is not an accurate representation of reality. One major detail that can be easily noticed is the difference  between the results before 4.29 and after 6.00 Gyrs.

\begin{figure}[ht!]
\plotone{M33_RotationCurveEvolution.png}
\caption{ This shows the simulated mean circular velocity as well as the analytical rotation curve of M33 within the Jacobi radii that are calculated for each pericenter and apocenter, as well as what time are calculated relative to the start of the simulation. As time passes, the simulated mean circular velocity becomes less like the analytical rotation curve of M33.
\label{fig:M33Rot}}
\end{figure}

\begin{figure}[ht!]
\plotone{M33_RotationCurve_Vphi.png}
\caption{ This shows the difference between the analytical rotation curve of M33 and its calculated mean  \(v_\phi\), as well as what time this rotation curve is calculated relative to the start of the simulation. The further from 0 this result is, the less M33 is like a dwarf irregular galaxy supported by rotation.
\label{fig:M33Vphi}}
\end{figure}


\section{Discussion}

It can be seen in Figure \ref{fig:M33Vphi} that, initially, the mean circular velocity is somewhat close to the analytical circular velocity, with a section of the plot of the difference between the two coming close to zero from ~3 kpc to ~12 kpc. However, as time passes, the difference between the analytical and mean circular velocity gets further from zero, which indicates that M33 is becoming less rotationally supported, in other words, less like a dwarf irregular. This becomes extremely apparent for the results 6.00 Gyrs and above, which can be seen by the large gap between the line for 4.29 and 6.00 Gyrs. The Milky Way and M31 merge during this time period, and it is likely that M33 lost a lot of mass during that merger as well, which could contribute to the gap that is seen.

In the paper written by Mayer et al., they state that galaxies closer to the Milky Way and M31 are mostly composed of dSphs and dEs, and dwarf irregular galaxies are found further away from the two galaxies \citep{Mayer2001}. The result that is described above can link tidal transformations of the satellite galaxy due to the host galaxy to the observation stated by Mayer et al. This means that galaxies closer to the Milky Way and M31 are tidally tranformed, which causes them to evolve into dSphs and dEs, while galaxies further away do not experience tidal transformations very strongly, so they remain dwarf irregulars. 

One of the uncertainties in my calculations is in the calculations of the various Jacobi Radii. The equation for the Jacobi Radius uses the mass of the satellite, which is supposed to change with time. However, I used the initial mass of M33 for each calculation, which would result in a larger Jacobi Radius.

\section{Conclusions}

Tidal transformations are the changes to a satellite galaxy due to the tides from a massive host galaxy, such as changes in the internal stellar kinematics of the satellite galaxy. Understanding these transformations is important because of their implications on galaxy evolution, in particular how satellite galaxies evolve due to their interactions with their host galaxy. I used the N-body simulation of the merging of the Milky and M31 that includes M33 that is used in \cite{MarelBesla2012} as a tool to explore this topic. I tried to determine how the merger of the Milky Way and M31 will affect the evolution of M33, in particular, whether M33 will evolve from a dwarf irregular galaxy to a dwarf spheroidal/elliptical galaxy due to tidal transformations from the merging of the Milky Way and M31.

I found that, as the Milky Way and M31 merge, the difference between the analytical and mean circular velocity of M33 gets further from zero, which indicates that M33 is becoming less rotationally supported, in other words, less like a dwarf irregular. This shows that there is a correlation between tidal transformations and a satellite galaxy's evolution from a dwarf irregular to a dwarf spheroidal/elliptical galaxy. This also agrees with my hypothesis that M33 would become less like a dwarf irregular.

A potential direction to further explore this topic would be to see if M33 truly evolves into a dwarf spheroidal/elliptical. This project showed that M33 evolves away from being a dwarf irregular galaxy, but does not provide much evidence for it becoming a dwarf spheroidal/elliptical galaxy. One way to provide this evidence would be to calculate the simulated surface mass density profile of M33's disk particles at each of the points of the simulation that I used and compare it to a Sersic profile \citep{Sersic1963} with n = 4, which is the same as a De Vaucouleurs profile. If M33's simulated surface mass density profile becomes more and more similar to the De Vaucouleurs profile, then it is becoming more and more like an elliptical galaxy.

\section{Acknowledgments}

I would like to acknowledge Dr. Gurtina Besla, who helped develop much of the code used in this project.

This work made use of the following software packages: \texttt{astropy} \citep{astropy:2013, astropy:2018, astropy:2022}, \texttt{matplotlib} \citep{Hunter:2007}, \texttt{numpy} \citep{numpy}, \texttt{python} \citep{python}, and \texttt{scipy} \citep{2020SciPy-NMeth, scipy_15366870}.

Software citation information aggregated using \texttt{\href{https://www.tomwagg.com/software-citation-station/}{The Software Citation Station}} \citep{software-citation-station-paper, software-citation-station-zenodo}.

\textit{We respectfully acknowledge the University of Arizona is on the land and territories of Indigenous peoples. Today, Arizona is home to 22 federally recognized tribes, with Tucson being home to the O’odham and the Yaqui. Committed to diversity and inclusion, the University strives to build sustainable relationships with sovereign Native Nations and Indigenous communities through education offerings, partnerships, and community service.}

\bibliography{sample7}{}
\bibliographystyle{aasjournalv7}

\end{document}
